\documentclass[12pt]{ctexart}
\usepackage[backend=biber,style=numeric,sorting=none, backref=true]{biblatex}  % 使用 APA 格式
% 设置为右上角标
% 方括号引用命令(行内)
\DeclareCiteCommand{\citeb}[\mkbibbrackets]
  {\usebibmacro{prenote}}
  {\usebibmacro{citeindex}%
   \usebibmacro{cite}}
  {\multicitedelim}
  {\usebibmacro{postnote}}

% 定义上标方括号格式的包装器
\newcommand{\mkbibsuperbrackets}[1]{\textsuperscript{[#1]}}

% 上标方括号引用命令(修正版)
\DeclareCiteCommand{\cites}[\mkbibsuperbrackets]
  {\iffieldundef{prenote}{}{\BibliographyWarning{Ignoring prenote argument}}%
   \iffieldundef{postnote}{}{\BibliographyWarning{Ignoring postnote argument}}}
  {\usebibmacro{citeindex}%
   \usebibmacro{cite}}
  {\multicitedelim}
  {}

\addbibresource{main.bib}  % 你的 .bib 文件
% 设置参考文献标题为中文
\DefineBibliographyStrings{english}{
  references = {参考文献}
}
% Figure
\usepackage{graphicx,xcolor}
% \usepackage{subfigure} % 子图支持
\usepackage{caption}
\usepackage{subcaption} % 子图支持
\DeclareCaptionLabelSeparator{mysep}{\hspace{2em}}
\captionsetup{labelsep=quad} % 图和表标题去掉冒号
\usepackage{pdfpages} % insert multi-page PDF
\usepackage{bm}
\usepackage{amssymb}

% Table
\usepackage{array}
\usepackage{booktabs}
\usepackage{multirow}
\usepackage{tabularx}
\usepackage{makecell}

% Formula
\usepackage{amsmath}

% Set
% \setcounter{secnumdepth}{3}
% \usepackage{xcolor}