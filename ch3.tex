% !TEX root = ch3.tex
%% ---DO NOT MODIFY BEGIN---
\ifx\allfiles\undefined
\documentclass[12pt]{ctexart}
\usepackage[backend=biber,style=numeric,sorting=none, backref=true]{biblatex}  % 使用 APA 格式
\addbibresource{main.bib}  % 你的 .bib 文件

% Figure
\usepackage{graphicx,xcolor}
% \usepackage{subfigure} % 子图支持
\usepackage{subcaption} % 子图支持
\usepackage{pdfpages} % insert multi-page PDF
\usepackage{bm}
\usepackage{amssymb}

% Table
\usepackage{array}
\usepackage{booktabs}
\usepackage{multirow}
\usepackage{tabularx}
\usepackage{makecell}

% Formula
\usepackage{amsmath}

% Set
% \setcounter{secnumdepth}{3}
% \usepackage{xcolor}
\begin{document}
% \makecover
\else
\fi
%% ---DO NOT MODIFY END---
%-------------------第3章------------------------

\section{FPV无人机穿戴式单手操作方法的提案}

本章提出了FPV无人机穿戴式单手操作方法的设计方案,并进行了原型和测试系统的设计。

\subsection{方案设计}\label{label:section2.1}

遥控系统的延迟模型可以用以下公式表示:
\begin{equation}
    \label{formula:2.1}
    y(t)=f(x(t-\tau)+\epsilon(t-\tau))
\end{equation}
其中:

\begin{itemize}
    \item $y(t)$ 是系统的输出,即无人机的行为,或响应
    \item $x(t)$ 是输入信号(例如:控制命令或传感器数据)
    \item $\tau$ 是系统的时间延迟(例如:控制器到无人机的信号传输延迟)
    \item $f(\cdot)$ 是控制与无人机之间动作对应的映射函数
    \item $\epsilon(t)$ 是噪声
\end{itemize}

对于以上的$x\xrightarrow{}y$的映射模型,本研究用两种方式实现。

考虑到四轴无人机是在三维空间中运动的4自由度物体,因此$y(t)$可以分解为以下4个基本动作:
\begin{itemize}
    \item 横滚(Roll):使无人机向横向倾斜(绕前后方向的纵轴旋转,实现左右移动)
    \item 俯仰(Pitch):使无人机向前后方向倾斜(绕左右方向的横轴旋转,实现前后移动)
    \item 偏航(Yaw):旋转无人机的朝向(绕水平面法线轴旋转),用于飞行中的方向转换
    \item 油门(Throttle):调整无人机的高度
\end{itemize}

\begin{figure}[t]
    \vspace{-20pt}
    \centering
    \includegraphics[width=.8\linewidth]{fig/3.1-6DOF-hand.png}
    \caption{手腕的自由度}
    \label{fig:6dof}

    \centering
    \begin{subfigure}[b]{0.3\textwidth}
        \includegraphics[width=\textwidth]{fig/3.2-Left_Right.png}
        \caption{偏航(Yaw)}
        \label{fig:yaw}
    \end{subfigure}
    \hfill
    \begin{subfigure}[b]{0.3\textwidth}
        \includegraphics[width=\textwidth]{fig/3.3-pronation_and_supination.png}
        \caption{横滚(Roll)}
        \label{fig:roll}
    \end{subfigure}
    \hfill
    \begin{subfigure}[b]{0.3\textwidth}
        \includegraphics[width=\textwidth]{fig/3.4-up_down.png}
        \caption{俯仰(Pitch)}
        \label{fig:pitch}
    \end{subfigure}
    \caption{对应旋转自由度的手势}
    \label{fig:three-guesture}
\end{figure}

\clearpage
因此,一种方法是将控制端的输入分别映射到这4个基本动作。

本研究实现一种类似于Yau等人\cites{yau2020subtle}的连续,细微手势。在该研究中,手势包括手腕旋转、手指接触、捏取和滑动。考虑到硬件实现的鲁棒性,本研究采用手腕旋转和手指接触·捏取动作作为手势。

对于手腕旋转手势,如图\ref{fig:6dof}所示,手腕旋转包含3个旋转自由度和3个平移自由度。其中对应3个旋转自由度的手势如图\ref{fig:three-guesture}所示。

对于手指接触·捏取动作手势,利用拇指与其他手指(例如:食指、中指)作为修饰输入,手指间的压力也可以作为额外输入信号。

基于以上的输入,建立一种$x\xrightarrow{}y$的映射,即可实现实时飞行中的位置机动控制。这部分的具体映射在第4章和第5章两次实验前进行说明。


另外一种方法类似于Kavarthapu\cites{kavarthapu2017hand}的基于深度学习,将手势与命令一一对应的方法。该研究所使用的是一种由基本手势进行排列组合成的通用手势。比如,基本手势包括手的上下左右摆动,组合手势就是由这四个组合成的,长度为n的序列。由于完成一个手势需要时间,将手势经过预测模型输出也需要时间,相对上一种方法实时性较差,本研究将用于对无人机施放离散命令。

\subsection{原型实现}\label{ssec:num1}

基于以上的控制理论,参考Covaciu\cites{covaciu2022control}等人的设计,进行了硬件原型的设计和实现。

Covaciu\cites{covaciu2022control}采用CPU(图\ref{fig:3.5-circuit}-6)与IMU (图\ref{fig:3.5-circuit}-5)的设计,本方案在此基础上加入薄膜压力传感
器(图\ref{fig:3.5-circuit}-(1-2)),和铜箔 (图\ref{fig:3.5-circuit}-(1-1),2,3)。电路的仿真示意图如图\ref{fig:4.2-1-circuit}所示。

\begin{figure}[t]
    \centering
    \includegraphics[width=1\linewidth]{fig/test.pdf}
    \caption{Enter Caption}
    \label{fig:placeholder}
\end{figure}

所使用的组件规格如表\ref{tab:hardware}所示。CPU和IMU参考已有研究\cites{covaciu2022control},采用带有无线传输的低功耗的ESP32,以及集成加速度计和陀螺仪的IMU,两者通过SCL和SDA双通道连接。为了扩展控制维度,本研究加入了小型薄膜压力传感器。传感器压力上限为22N,其阻值-压力关系是一个近似反比例曲线。由于实际测量该曲线较为困难,采用了生产厂家提供的经验曲线,如图\ref{fig:placeholder}。参考一项关于指侧捏力的研究\cites{呼慧敏2022中国成年人指侧捏力测量研究},我们确定人在日常使用设备时捏力在5N至50N之间,对应于图中500g以上压力值,该部分近似线性。假定线性部分的阻值-压力关系如下:
$$
R = kF+R_0, F<22
$$
根据经验曲线,$k=0.29\pm 0.07$,偏置$R_0=8.9\pm1$。通过如图\ref{fig:3.5-circuit}所示的电阻R1分压电路,得到P1点的电位:
$$
V_{p1} = \frac{3.3R_1}{kF+R_0+R_1}
$$
综合$R_0$的大小和0域附近的电阻变化,我们取$R_1=10k\Omega$。

考虑到同时具有柔韧性和优良的导电性的材料,这里使用铜箔分别附在大拇指,中指和无名指外侧,分别连接到图\ref{fig:3.5-circuit}的(1-1),2和3。CPU通过将2和3端口软件设定为上拉输入模式。

控制器原型实物如图\ref{fig:4.2-2-prototype}所示。CPU和IMU通过PCB电气连接并附着在手掌背部,用于手的姿势检测。薄膜压力传感器附在大拇指里侧,检测大拇指与其它手指之间的捏力。铜箔检测指间的触碰。

\begin{table}[b]
    \centering
    \caption{\label{tab:hardware}组件规格}
    \begin{tabular}{>{\raggedright\arraybackslash}p{2.5cm} p{4cm} p{4cm}}
        \toprule
        \textbf{组件} & \textbf{型号}        & \textbf{特征}                                                  \\
        \midrule
        CPU         & ESP32-WROOM-32E    & 集成2.4 GHz Wi-Fi和蓝牙双模,低功耗                                   \\ \midrule
        IMU       & BNO08X             & 集成三轴加速度计、三轴陀螺仪和三轴磁力计,内置校准和自适应功能,实现高精度和低延迟                 \\ \midrule
        薄膜压力传感器     & DF9-16,2×2.5×0.1cm & 小型尺寸,适合穿戴设备                                                \\ \midrule
        铜箔          & 0.05mm             & 具有柔韧性和优良的导电性                                               \\ \midrule
        电源          & 5V-DC              & 标准电源                                                         \\
        \bottomrule
    \end{tabular}
\end{table}

\begin{figure}
    \centering
    \includegraphics[width=1\linewidth]{fig/3.5-circuit.pdf}
    \caption{电路示意图}
    \label{fig:3.5-circuit}
\end{figure}

\begin{figure}
    \vspace{-10pt}
    \includegraphics[width=.7\linewidth]{ fig/3.6-circuit-real.pdf}
    \caption{原型电路图:}
    \label{fig:4.2-1-circuit}
    \vspace{-20pt}
\end{figure}



\begin{figure}[htbp]
    \centering
    \includegraphics[width=.7\linewidth]{ fig/3.6-prototype.png}
    \caption{穿戴式控制器原型}
    \label{fig:4.2-2-prototype}
\end{figure}

\clearpage

\subsection{仿真器实现}

使用Unity3D引擎构建无人机仿真器系统。

四轴无人机使用Unity Asset Store提供的模型(图\ref{fig:4.2-4-vision_mode})和基本的运动控制系统。实体的碰撞模型和尺寸如图\ref{fig:4.2-3-collision_model}所示。无人机的质量为$m=10$,角阻力(Angular Drag)为0.05。单位均为Unity内部单位。

\begin{figure}[htbp]
    \centering
    \includegraphics[width=.9\linewidth]{ fig/3.7-vision_mode.png}
    \caption{四轴无人机可视化模型}
    \label{fig:4.2-4-vision_mode}

    \includegraphics[width=.9\linewidth]{ fig/3.8-collision_model.pdf}
    \caption{四轴无人机实体碰撞模型(单位:英尺)}
    \label{fig:4.2-3-collision_model}
\end{figure}

下面说明仿真器中无人机的运动控制。无人机的三维运动通过实现\ref{label:section2.1}节所述的4个动作来实现。

\begin{itemize}
    \item Yaw动作的输入与响应关系如下所示:
          \begin{equation}
              \begin{aligned}
                  \theta_{\text {target }}(t) & =\theta_{\text {target }}(t-\Delta t)+u_y(t-\Delta t) \cdot v_{\max }, \\
                  v(t)                        & =\frac{\theta_{\text {target }}(t)-\theta(t-\Delta t)}{\tau},          \\
                  \theta(t)                   & =\theta(t-\Delta t)+v(t) \cdot \Delta t .
              \end{aligned}
          \end{equation}
          其中:
          $\theta_{\text {target }}(t)$是最终目标角度,作为独立变量存储和更新,$\theta(t)$是Unity画面实际更新的角度。两者都是相对于世界坐标系的垂直轴角度。$v(t)$是平滑更新函数。$u_y(t)$是Yaw的输入(例如遥控器的摇杆值)。$v_{\max }$是超参数,表示最大Yaw速度。$\tau$是平滑时间常数,这里设为0.25(s)。$\Delta t$是时间步长,指Unity的帧更新间隔。
\end{itemize}

\begin{itemize}
    \item Roll动作使用Unity的物理系统,通过力的仿真进行控制:
          \begin{equation}
              \begin{gathered}
                  \vec{F}_{\text {roll }}(t)=u_r(t-\Delta t) \cdot F_{\text {side }} \cdot \vec{i},
              \end{gathered}
          \end{equation}
          其中,$u_r(t-\Delta t)$是输入,$\Vec{i}$是单位向量,始终指向无人机的左侧方,$F_{\text {side }}$是用于调整输入尺度的超参数。
\end{itemize}

\begin{itemize}
    \item Pitch动作的控制如下所示:
          \begin{equation}
              \begin{gathered}
                  \vec{F}_{\text {pitch }}(t)=u_p(t-\Delta t) \cdot F_{\text {forward }} \cdot \vec{j},
              \end{gathered}
          \end{equation}
          其中,$u_p$是输入,$\Vec{j}$是单位向量,始终指向无人机的正前方,$F_{\text {forward }}$是用于调整输入尺度的超参数。
\end{itemize}

\begin{itemize}
    \item Throttle动作的控制如下所示:
          \begin{equation}
              \begin{aligned}
                  F_{\text{input}}(t) =
                  \begin{cases}
                      u_t(t-\Delta t) \cdot F_{\text{upward}},   & u_t > 0, \\
                      u_t(t-\Delta t) \cdot F_{\text{downward}}, & u_t < 0, \\
                      0,                                         & u_t = 0,
                  \end{cases} \\
                  \Vec{F}_{\text{up}}(t) = (m \cdot g + F_{\text{input}})(t) \cdot \Vec{k}
              \end{aligned}
          \end{equation}

          其中,$u_t$是输入,$F_{\text {upward }}$是超参数,$m$是无人机质量,$g$是重力加速度(这里为$9.81 m/s^2$)。$\Vec{k}$是垂直向上的单位向量。需要注意的是,与Roll和Pitch动作不同,Throttle的上升和下降所施加的力是不同的。这是因为实际无人机在快速下降时会因涡环状态(Vortex Ring State)导致升力下降,可能失去控制。为避免这个问题,采用了抑制下降速度的设计。本仿真器也再现了这一特性。

\end{itemize}

\subsection{小结}
本章提出了面向FPV无人机操作的动态手势交互界面,通过手腕旋转和手指接触等动作实现了高效且直观的控制。基于这一交互界面,采用低功耗、高精度的传感器和无线通信技术,设计并实现了穿戴式控制器原型。此外,使用Unity3D引擎构建了无人机仿真器系统。这为今后的界面可用性研究和优化奠定了基础。
\clearpage
%% ---DO NOT MODIFY BEGIN---
\ifx\allfiles\undefined
\end{document}
\fi
%% ---DO NOT MODIFY END---