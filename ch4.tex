% !TEX root = ch4.tex
%% ---DO NOT MODIFY BEGIN---
\ifx\allfiles\undefined
\documentclass[12pt]{ctexart}
\usepackage[backend=biber,style=numeric,sorting=none, backref=true]{biblatex}  % 使用 APA 格式
\addbibresource{main.bib}  % 你的 .bib 文件

% Figure
\usepackage{graphicx,xcolor}
% \usepackage{subfigure} % 子图支持
\usepackage{subcaption} % 子图支持
\usepackage{pdfpages} % insert multi-page PDF
\usepackage{bm}
\usepackage{amssymb}

% Table
\usepackage{array}
\usepackage{booktabs}
\usepackage{multirow}
\usepackage{tabularx}
\usepackage{makecell}

% Formula
\usepackage{amsmath}

% Set
% \setcounter{secnumdepth}{3}
% \usepackage{xcolor}
\begin{document}
% \makecover
\else
\fi
%% ---DO NOT MODIFY END---
%-------------------第4章------------------------

\section{与GamePad的对比实验}
 
本章进行探索性实验。为验证所提出的穿戴式单手控制器的有用性,设计了输入与无人机动作的映射关系。在模拟器中实现后,与GamePad进行对比实验。GamePad是目前最广泛使用的传统双手遥控器,通过本实验旨在确认本方案的潜力,发现系统整体的不足,并探索各方面的改进方向。

首先说明本次实验中使用的输入与无人机动作的映射关系,然后描述实验使用的关卡和问卷,接着展示实验流程,最后给出实验结果并进行讨论。

\subsection{动作映射}

本实验使用所提出的原型和Logicool Gamepad F310(图\ref{5.1-1-logicool})作为输入设备。

表\ref{tab:gesture}显示了所提出原型的输入(手势)与无人机动作的映射关系,“无人机输入”对应于3.2节各个动作的输入。在此映射中,当拇指与食指接触时,激活“可前进”状态,在该状态下,通过手部的Yaw、Pitch、用力捏取这3个动作的组合实现无人机在三维空间中的移动;当拇指与中指接触时,激活“可后退”状态,在该状态下,通过手部的Yaw和用力捏取动作来控制无人机的后退与后退偏移。这种设计有以下原因:

\begin{table}[b]
    \centering
    \caption{\label{tab:gesture}动作映射}
    \begin{tabular}{>{\centering\arraybackslash}p{1.5cm} >{\centering\arraybackslash}p{3.8cm} >{\centering\arraybackslash}p{2cm} >{\centering\arraybackslash}p{3cm}}
        \toprule
        \textbf{状态} & \textbf{手势}           & \textbf{无人机输入} & \textbf{无人机动作} \\
        \midrule
        \multirow{3}{3cm}{可前进}
                      & 手腕左右旋转 (Yaw)       & \(u_y\)        & 视角旋转 (Yaw)    \\
                      & 手腕上下运动 (Pitch)     & \(u_t\)        & 升降 (Throttle) \\
                      & 用力捏取               & \(u_p\)        & 前进            \\
        \midrule
        \multirow{3}{3cm}{可后退}
                      & 手腕左右旋转 (Yaw)       & \(u_r\)        & 横移 (Roll)     \\
                      & 手腕上下旋转 (Pitch)     & 无               & 无             \\
                      & 用力捏取               & \(u_p\)        & 后退            \\
        \bottomrule
    \end{tabular}
\end{table}

\begin{itemize}
    \item 实际的直升机飞行中,可以单独或组合使用前进+方向转换+升降的基本操作。
    \item 在大多数场景中,无人机的前进动作比后退动作更加频繁。
\end{itemize}

GamePad使用两个摇杆作为位置控制的输入。左摇杆控制水平面移动(Pitch和Roll),右摇杆控制视角旋转和升降(Throttle和Yaw)。

需要注意的是,本方案使用的手腕旋转手势类似于鼠标或触摸板的位置映射控制(Position Mapping Control),将手腕运动的幅度按一定比例映射为无人机移动的距离。而GamePad使用方向触发控制(Direction Trigger Control),摇杆的方向对应移动方向,摇杆从零点的偏移量对应速度。

\begin{figure}[htbp]
    \centering
    \includegraphics[width=.7\linewidth]{fig/4.1-logicool.png}
    \caption{\label{5.1-1-logicool}Logicool Gamepad F310(https://resource.logitechg.com/)}
\end{figure}

\subsection{实验关卡和问卷}\label{ssec:coures_and_qustionnaire}

本次实验中,参与者需要完成在无人机模拟器中构建的实验关卡。参与者首先在练习关卡上进行练习,然后通过4条难度不同的关卡(图\ref{fig:5.2-1-4courses})。操作者需要逐个通过"门",每个门的尺寸为$4 \times 4$(英尺),所有门都设置为可碰撞对象。

\begin{figure}[htbp]
    \centering
    \includegraphics[width=\linewidth]{fig/4.2-4courses.pdf}
    \caption{\label{fig:5.2-1-4courses}四条不同难度(NOO)的路线:红色方框为“门”,黑色线条为路径}
\end{figure}

使用操作数(Number of operations, NOO)来表示路线的难度。NOO的计算方法如下:

\begin{itemize}
    \item 基于关卡的路径,计算其最低所需基本操作数(Yaw, Pitch, Roll, Throttle)。
    \item 两个连续门之间的NOO由必要最小操作数决定。
    \item 对于2个或多个平行门(比如图\ref{fig:5.2-1-4courses}中Level 2的6\~9号),虽然无必要操作,但操作者仍需保持稳定飞行,所以无论通过门数量有多少NOO均记录为1。
\end{itemize}

例如,图\ref{fig:5.2-1-4courses}所示Level 2路线的NOO计算如下:

\begin{itemize}
    \item 门\textcircled{1} $\Rightarrow$门\textcircled{2}: 1(前进)
    \item 门\textcircled{2} $\Rightarrow$门\textcircled{3}: 2(上升+前进)
    \item 门\textcircled{3} $\Rightarrow$门\textcircled{4}: 2(下降+前进)
    \item 门\textcircled{4} $\Rightarrow$门\textcircled{5}: 3(视角旋转+上升+前进)
    \item 门\textcircled{5} $\Rightarrow$门\textcircled{6}: 3(视角旋转+下降+前进)
    \item 门\textcircled{6} $\Rightarrow$门\textcircled{9}: 1(前进)
\end{itemize}

问卷使用Di等人\cite{di2022natural}采用的两个问卷。一个是关于身体认知的问卷(Embodied Cognition(EC)),另一个是关于VR体验的问卷(Virtual Experience(VE))(参见表\ref{tab:Q1}和表\ref{tab:Q2})。将其翻译为中文,排除与本次实验无关的第16项,问卷评价形式采用7级评价。

问卷EC主要评价以下3个方面:
(1) 自我定位感(Self-location):身体刺激的感知和位置的认知(问题项目 6, 7, 9, 14, 15, 20)。
(2) 身体所有感(Body-ownership):身体所有感的意识(问题项目 8, 11, 12, 13, 15, 17, 18, 19)。
(3) 动作主体感(Agency):对动作的控制感(问题项目 1, 2, 3, 4, 5, 10)。

\begin{table}[]
    \centering
    \caption{问卷EC(Embodied Cognition)}
    \label{tab:Q1}
    \begin{tabular}{l}
        1. 我感觉我在操控无人机                    \\
        2. 无人机按照我的指示行动                   \\
        3. 无人机按我想的那样运动                   \\
        4. 我感觉自己对看到的动作负有责任               \\
        5. 无人机在自己运动                      \\
        6. 我感觉身处虚拟环境中                    \\
        7. 我感觉自己的视野变成了无人机的视角            \\
        8. 我感觉变成了无人机                     \\
        9. 我有与无人机同步运动的感觉                 \\
        10. 我感觉自己的手在操作无人机               \\
        11. 我感觉自己与自己的身体分离               \\
        12. 我感觉有另一个身体                    \\
        13. 我感觉无人机的动作受到我的动作影响          \\
        14. 我在无人机起飞时感到上升感               \\
        15. 我感觉像无人机一样在漂浮                \\
        (16). 我在无人机着陆时感到下降感             \\
        17. 我感觉以某种形式与无人机相连              \\
        18. 我感觉无人机是自己的一部分               \\
        19. 我感觉无人机就是我自己                  \\
        20. 我感觉与无人机在同一个地方
    \end{tabular}
\end{table}

\begin{table}[]
    \centering
    \caption{问卷VE(Virtual Experience)}
    \label{tab:Q2}
    \begin{tabular}{l}
        1. 这种控制方法反应迅速    \\
        2. 这种控制方法很舒适     \\
        3. 这种控制方法很简单     \\
        4. 这种控制方法没有错误    \\
        5. 这种控制方法容易学习    \\
        6. 这种控制方法很自然     \\
        7. 这种控制方法很有用     \\
        8. 我还想再使用这种控制方法
    \end{tabular}
\end{table}

\subsection{实验流程}
共有24名参与者(22~28岁;男性23名,女性1名;全部为右利手)参加了本实验。所有人都没有无人机操作经验。实验采用被试间设计(between subject design),分为2组。每组12人,一组使用本方案的控制器(N组),另一组使用GamePad(T组)。两组都通过PC显示器进行FPV操作。

实验由4个阶段组成(S1:介绍,S2:练习,S3:主实验,S4:对比调查)。

\begin{itemize}
    \item \textbf{S1} 向参与者说明实验目的、流程和操作方法,获得知情同意。
    \item \textbf{S2} 参与者在练习路线上熟悉控制器操作。
    \item \textbf{S3} 参与者按难度顺序各通过图\ref{fig:5.2-1-4courses}所示的4条路线1次,共24名$\times$ 4路线$\times$ 1轮 = 96次试验。
    \item \textbf{S4} 参与者完成问卷EC和问卷VE。
\end{itemize}

% \clearpage

\subsection{结果}\label{study1_result}

由于每个"门"都设置为可碰撞对象,记录了碰撞次数。各组的平均碰撞次数如图\ref{fig:5.4-1-exp1_time_and_collision}(左)所示。平均通过时间如图\ref{fig:5.4-1-exp1_time_and_collision}(右)所示。N组的通过时间和碰撞次数都高于T组。

图~\ref{fig:5.4-1-result-Q1Q2}显示了问卷EC和问卷VE各方面N组与T组的平均得分。在所有3个方面都未见显著差异($p<.05$),但N组在身体所有感(Body-ownership)方面呈现较高趋势。

\begin{figure}[t]
    \centering
    \begin{subfigure}[b]{0.48\textwidth}
        \includegraphics[width=\textwidth]{fig/4.3-line_chart_collision_count_trend.png}
    \end{subfigure}
    % \hfill
    \begin{subfigure}[b]{0.48\textwidth}
        \includegraphics[width=\textwidth]{fig/4.4-line_chart_flight_time_trend.png}
    \end{subfigure}
    \caption{\label{fig:5.4-1-exp1_time_and_collision}平均碰撞次数和平均通过时间}
\end{figure}

\begin{figure}[t]
    \centering
    \includegraphics[width=\linewidth]{fig/4.6-questionnaire_1and2_t-test.png}
    \caption{\label{fig:5.4-1-result-Q1Q2}“身体认知”结果(左)“VR体验”结果(右)}
\end{figure}

\subsection{讨论}

问卷各方面的结果均未见显著差异。从实验结果来看,自我定位感和身体所有感呈现良好趋势。另一方面,整体上未见显著优势。这可能与参与者平时使用GamePad的习惯有关。习惯GamePad的人倾向于GamePad。而平时不使用GamePad的人(本次实验中有两人)回答更喜欢本方案。作为解决此问题的对策,可以将比较对象从一般熟知的Gamepad替换为DJI Avata等单手控制器,以减少影响。与DJI Avata的比较实验将在第五章中述及。

此外,我们注意到实验中的参与者在前进的同时进行转向时失误率较高。我们首先考虑了手势-无人机动作映射表\ref{tab:gesture}的设计合理性。本实验的GamePad中,前进(Pith)放在左摇杆,转向(Yaw)放在右摇杆,采用的是一种前进(Pith)与转向(Yaw)分离的方案。而本方案将两个动作整合在一起,前进(Pith)的同时可进行转向(Yaw)。对于这两种方案的合理性,需进一步的实验来验证。另外,我们也考虑了模拟器运动学的不足。本实验使用的无人机仿真器采用了简易的三维运动控制,也就是通过方向+位移来控制,未充分考虑实际的运动学。由于空中没有摩擦力,无人机在方向转换时缺乏横向向心力的问题。需要从运动学角度补偿方向转换时的力,改进仿真。

\subsection{小结}

本次实验中提出了手势与无人机动作的映射关系,并在穿戴设备原型中实现。本实验进行了与GamePad的比较实验。结果显示,在自我定位感和身体所有感方面呈现良好趋势,但在平均碰撞数、通过时间、主观评价方面,本方案的控制器不如GamePad。总结了以下因素和改进方案:

\begin{itemize}
    \item 由于参与者习惯GamePad,倾向于喜欢熟悉的方式。需要考虑与先行研究的比较实验。
    \item 输入-无人机动作映射可能不自然。作为改进方案,需要比较多种映射方案。
    \item 无人机方向转换时向心力不足。需要进行运动学修正,改进方向转换动作。
\end{itemize}

实验表明,本方案在沉浸感方面具有潜力。同时,为进一步研究提供了改进方向。

%% ---DO NOT MODIFY BEGIN---
\ifx\allfiles\undefined
\end{document}
\fi
%% ---DO NOT MODIFY END---