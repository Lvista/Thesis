% !TEX root = front_pages.tex
%% ---DO NOT MODIFY BEGIN---
\ifx\allfiles\undefined
\documentclass[12pt]{ctexart}
\usepackage[backend=biber,style=numeric,sorting=none, backref=true]{biblatex}
\usepackage{graphicx,xcolor}
\usepackage{caption}
\usepackage{subcaption} % 子图支持
\usepackage{chngcntr}
\usepackage{pdfpages} % insert multi-page PDF
\usepackage{bm}
\usepackage{amssymb}
\usepackage{array}
\usepackage{booktabs}
\usepackage{multirow}
\usepackage{tabularx}
\usepackage{makecell}
\usepackage{amsmath}

% 设置章节标题格式(针对section)
\ctexset{
    section = {
        format = \centering\heiti\Large,
        name = {第,章},
        number = \arabic{section},
        aftername = \hspace{0.5em}
    },
    subsection = {
        format = \raggedright\heiti\large,
        name = {},
        number = \arabic{section}.\arabic{subsection},
        aftername = \hspace{0.5em}
    }
}

% 设置为右上角标
% 方括号引用命令(行内)
\DeclareCiteCommand{\citeb}[\mkbibbrackets]
  {\usebibmacro{prenote}}
  {\usebibmacro{citeindex}%
   \usebibmacro{cite}}
  {\multicitedelim}
  {\usebibmacro{postnote}}

% 定义上标方括号格式的包装器
\newcommand{\mkbibsuperbrackets}[1]{\textsuperscript{[#1]}}

% 上标方括号引用命令(修正版)
\DeclareCiteCommand{\cites}[\mkbibsuperbrackets]
  {\iffieldundef{prenote}{}{\BibliographyWarning{Ignoring prenote argument}}%
   \iffieldundef{postnote}{}{\BibliographyWarning{Ignoring postnote argument}}}
  {\usebibmacro{citeindex}%
   \usebibmacro{cite}}
  {\multicitedelim}
  {}

\addbibresource{main.bib}  % 你的 .bib 文件

% 设置参考文献标题为中文
\DefineBibliographyStrings{english}{
  references = {参考文献}
}

% 图注设置
\captionsetup[figure]{
    labelsep=quad,
    name=图,
    format=plain,
    font=small,
    labelfont=bf
}

% 表注设置
\captionsetup[table]{
    labelsep=quad,
    name=表,
    format=plain,
    font=small,
    labelfont=bf
}

% 图和表按节编号
\counterwithin{figure}{section}
\counterwithin{table}{section}

% 如果需要按章节编号(如果使用chapter),但ctexart没有chapter
% 如果确实需要章节,建议改用ctexbook文档类
% \renewcommand{\thefigure}{\thesection.\arabic{figure}}
% \renewcommand{\thetable}{\thesection.\arabic{table}}

\begin{document}
% \makecover
\else
\fi
%% ---DO NOT MODIFY END---
%-------------------第0章------------------------

\newpage
\thispagestyle{empty}
\begin{center}
    \vspace*{30pt}
    %タイトル
    {\huge 基于IMU的四旋翼无人机远程遥控方法研究}

    \vspace{10pt}
    %所属
    {\large 微系统与测控技术}

    \vspace{10pt}
    %名前
    {\large 程 智超}
    {\large 2025年10月1日}
    \clearpage
    \renewcommand{\abstractname}{\large 摘要}
    \begin{abstract}
        \vspace{20pt}
        {\normalsize
随着无人机技术的快速发展,其在民用领域的应用日益广泛,对高效、直观的人机交互方式提出了更高要求。传统双手遥控器(GamePad)操作复杂、占用双手,限制了无人机在移动场景下的使用体验。针对FPV无人机单手控制中“高维度控制需求”与“有限输入带宽”之间的核心矛盾,本研究攻克了输入维度不足、绝对映射不精确、沉浸感与操控性难以兼得等关键难题,提出一种面向第一人称视角(FPV)无人机的可穿戴式单手手势控制系统。

本文创新性地设计了基于多传感器融合的硬件原型,提出了“三状态机”模态切换模型,通过拇指与不同手指的接触触发,将有限的手势输入扩展为多维控制指令,有效解决了输入维度瓶颈;采用相对定位映射逻辑,将手腕旋转变化量映射为无人机速度指令,显著提升了控制的精确性与抗疲劳性。

通过构建Unity3D仿真平台并开展系统性的用户实验,本文验证了所提系统的有效性。实验结果表明,在头戴显示器(HMD)环境下,该系统在沉浸感(尤其是身体所有感与自我定位感)上显著优于传统手持控制器(如DJI Avata2类方案),在操作稳定性与精度上优于基于IMU倾斜映射的模仿控制方案,实现了沉浸感与操控性的有效平衡。

另外,以上维度问题只考虑了飞行中的机动控制,没考虑到起飞,降落等单独命令。因此,进一步对基于深度学习的手势命令识别进行研究。针对变长序列手势识别的一个挑战是,不同组合长度的手势序列需要单独收集对应的数据,这加大了数据收集阶段的负担。因此本文探索了基于CTC算法的手势识别模型。

为探明CTC算法的变长序列的泛化性能,进行了预实验。实验对比4种模型。其中GRU-CTC模型表现最优,准确率96\%,参数量仅14.4K。进一步改进一种回归算法模型,改进后的CTC算法模型相比原模型,在训练集长度上准确率从88\%提升至97\%,并解决了原模型无法识别非训练集长度序列的问题。

本研究不仅为FPV无人机提供了一种自然、高效的单手交互解决方案,更从硬件设计、交互逻辑、实证评估到算法层面,系统性地解决了该领域的一系列核心挑战,为人机交互设计与无人机技术的融合发展提供了重要的理论依据与工程实践参考。
        }

            %キーワード 3~5つ
            {\noindent\normalsize \textbf{关键词:} 无人机遥控;手势识别;可穿戴设备;第一人称视角;\\
            \hspace{2em}人机交互;深度学习}

    \end{abstract}
    %提出日
\end{center}


%% ---DO NOT MODIFY BEGIN---
\ifx\allfiles\undefined
\end{document}
\fi
%% ---DO NOT MODIFY END---

