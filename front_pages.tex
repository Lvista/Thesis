% !TEX root = front_pages.tex
%% ---DO NOT MODIFY BEGIN---
\ifx\allfiles\undefined
\documentclass[12pt]{ctexart}
\usepackage[backend=biber,style=numeric,sorting=none, backref=true]{biblatex}
\usepackage{graphicx,xcolor}
\usepackage{caption}
\usepackage{subcaption} % 子图支持
\usepackage{chngcntr}
\usepackage{pdfpages} % insert multi-page PDF
\usepackage{bm}
\usepackage{amssymb}
\usepackage{array}
\usepackage{booktabs}
\usepackage{multirow}
\usepackage{tabularx}
\usepackage{makecell}
\usepackage{amsmath}

% 设置章节标题格式(针对section)
\ctexset{
    section = {
        format = \centering\heiti\Large,
        name = {第,章},
        number = \arabic{section},
        aftername = \hspace{0.5em}
    },
    subsection = {
        format = \raggedright\heiti\large,
        name = {},
        number = \arabic{section}.\arabic{subsection},
        aftername = \hspace{0.5em}
    }
}

% 设置为右上角标
% 方括号引用命令(行内)
\DeclareCiteCommand{\citeb}[\mkbibbrackets]
  {\usebibmacro{prenote}}
  {\usebibmacro{citeindex}%
   \usebibmacro{cite}}
  {\multicitedelim}
  {\usebibmacro{postnote}}

% 定义上标方括号格式的包装器
\newcommand{\mkbibsuperbrackets}[1]{\textsuperscript{[#1]}}

% 上标方括号引用命令(修正版)
\DeclareCiteCommand{\cites}[\mkbibsuperbrackets]
  {\iffieldundef{prenote}{}{\BibliographyWarning{Ignoring prenote argument}}%
   \iffieldundef{postnote}{}{\BibliographyWarning{Ignoring postnote argument}}}
  {\usebibmacro{citeindex}%
   \usebibmacro{cite}}
  {\multicitedelim}
  {}

\addbibresource{main.bib}  % 你的 .bib 文件

% 设置参考文献标题为中文
\DefineBibliographyStrings{english}{
  references = {参考文献}
}

% 图注设置
\captionsetup[figure]{
    labelsep=quad,
    name=图,
    format=plain,
    font=small,
    labelfont=bf
}

% 表注设置
\captionsetup[table]{
    labelsep=quad,
    name=表,
    format=plain,
    font=small,
    labelfont=bf
}

% 图和表按节编号
\counterwithin{figure}{section}
\counterwithin{table}{section}

% 如果需要按章节编号(如果使用chapter),但ctexart没有chapter
% 如果确实需要章节,建议改用ctexbook文档类
% \renewcommand{\thefigure}{\thesection.\arabic{figure}}
% \renewcommand{\thetable}{\thesection.\arabic{table}}

\begin{document}
% \makecover
\else
\fi
%% ---DO NOT MODIFY END---
%-------------------第0章------------------------

\newpage
\thispagestyle{empty}
\begin{center}
    \vspace*{30pt}
    %タイトル
    {\huge 基于IMU的四旋翼无人机远程遥控方法研究}

    \vspace{10pt}
    %所属
    {\large 微系统与测控技术}

    \vspace{10pt}
    %名前
    {\large 程 智超}
    {\large 2025年10月1日}
    \clearpage
    \renewcommand{\abstractname}{\large 摘要}
    \begin{abstract}
        \vspace{20pt}
        {\normalsize
随着第一人称视角(FPV)无人机的快速发展,传统双手遥控器存在体积大、操作不直观等问题。本研究针对FPV无人机控制场景,提出了一种基于手势识别的可穿戴式单手操作方法。

由于已有的基于ESP32和IMU(惯性测量单元)方案在可控维度,易用性上的不足,我们提出在硬件上增加了压力传感器和接触传感的方案。在手势映射方案上,提出通过手腕的三个旋转自由度结合手指接触状态,实现对无人机运动的精确控制。与传统方向触发控制相比,位置映射控制类似鼠标操作,具有更高的精度和稳定性。

为验证方案有效性,使用Unity3D构建了仿真系统,设计了基于操作数(NOO)的难度评估方法,邀请了数名志愿者,在仿真系统中完成不同难度的越障关卡,并完成两份问卷,在主观和客观上对遥控方案进行评估。共进行了三组对比实验:与GamePad相比,在主观评价上,本方案在身体所有感方面表现略高;与已有的基于ESP32和IMU的方案对比,主观评价上,本方案在沉浸感、响应速度和控制精度方面显著高于前者($p <.05$);与DJI Avata2的对比,主观评价上本方案在沉浸感方面显著优优于前者($p <.002$)。

此外,为了扩展遥控命令的维度,我们考虑了序列手势的深度学习模型,并着重于变长序列的泛化性能,因此将训练集和测试集分为不同序列长度。我们验证了基于CTC算法的模型可行性,并就边缘设备计算资源受限的场景,对比了不同参数下4个模型的性能表现。其中GRU-CTC模型以96\%的平均准确率,14.4K参数量相对最优。

进一步的,我们将已有研究的回归算法模型修改为CTC算法模型,对比二者性能,改进的CTC模型在3个组合长度的数据上达到平均93\%的准确率,其中在训练集长度上,改进模型以97\%的准确率高于原模型的88\%。而对于非训练集长度,原模型无法识别,改进模型正常识别。

我们的可穿戴遥控硬件设计配合位置映射控制为无人机遥控提供了新的思路。作为补充,探索了序列手势的深度学习模型,证明了基于CTC算法的模型的变长序列的泛化性能,进行的不同模型复杂度的性能对比实验还为边缘手势识别设备提供了参考。此外提出的改进模型对比也展示了回归算法算法和CTC算法之间的区别,为相关研究提供了参考。
        }
        \begin{center}
            %キーワード 3~5つ
            {\normalsize \textbf{关键词:} 无人机遥控;手势识别;可穿戴设备;第一人称视角;人机交互;深度学习}
        \end{center}
    \end{abstract}
    %提出日

\end{center}

%% ---DO NOT MODIFY BEGIN---
\ifx\allfiles\undefined
\end{document}
\fi
%% ---DO NOT MODIFY END---

