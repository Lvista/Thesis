% !TEX root = ch7.tex
%% ---DO NOT MODIFY BEGIN---
\ifx\allfiles\undefined
\documentclass[12pt]{ctexart}
\usepackage[backend=biber,style=numeric,sorting=none, backref=true]{biblatex}
\usepackage{graphicx,xcolor}
\usepackage{caption}
\usepackage{subcaption} % 子图支持
\usepackage{chngcntr}
\usepackage{pdfpages} % insert multi-page PDF
\usepackage{bm}
\usepackage{amssymb}
\usepackage{array}
\usepackage{booktabs}
\usepackage{multirow}
\usepackage{tabularx}
\usepackage{makecell}
\usepackage{amsmath}

% 设置章节标题格式(针对section)
\ctexset{
    section = {
        format = \centering\heiti\Large,
        name = {第,章},
        number = \arabic{section},
        aftername = \hspace{0.5em}
    },
    subsection = {
        format = \raggedright\heiti\large,
        name = {},
        number = \arabic{section}.\arabic{subsection},
        aftername = \hspace{0.5em}
    }
}

% 设置为右上角标
% 方括号引用命令(行内)
\DeclareCiteCommand{\citeb}[\mkbibbrackets]
  {\usebibmacro{prenote}}
  {\usebibmacro{citeindex}%
   \usebibmacro{cite}}
  {\multicitedelim}
  {\usebibmacro{postnote}}

% 定义上标方括号格式的包装器
\newcommand{\mkbibsuperbrackets}[1]{\textsuperscript{[#1]}}

% 上标方括号引用命令(修正版)
\DeclareCiteCommand{\cites}[\mkbibsuperbrackets]
  {\iffieldundef{prenote}{}{\BibliographyWarning{Ignoring prenote argument}}%
   \iffieldundef{postnote}{}{\BibliographyWarning{Ignoring postnote argument}}}
  {\usebibmacro{citeindex}%
   \usebibmacro{cite}}
  {\multicitedelim}
  {}

\addbibresource{main.bib}  % 你的 .bib 文件

% 设置参考文献标题为中文
\DefineBibliographyStrings{english}{
  references = {参考文献}
}

% 图注设置
\captionsetup[figure]{
    labelsep=quad,
    name=图,
    format=plain,
    font=small,
    labelfont=bf
}

% 表注设置
\captionsetup[table]{
    labelsep=quad,
    name=表,
    format=plain,
    font=small,
    labelfont=bf
}

% 图和表按节编号
\counterwithin{figure}{section}
\counterwithin{table}{section}

% 如果需要按章节编号(如果使用chapter),但ctexart没有chapter
% 如果确实需要章节,建议改用ctexbook文档类
% \renewcommand{\thefigure}{\thesection.\arabic{figure}}
% \renewcommand{\thetable}{\thesection.\arabic{table}}

\begin{document}
% \makecover
\else
\fi
%% ---DO NOT MODIFY END---
%-------------------第7章------------------------

\section{结论}
本论文对无人机的控制模式进行了研究。提出了基于手腕旋转的手势操作方案,该手势能够实现高精度的第一人称视角无人机远程操作。将此方法实现到可穿戴设备原型中,并与传统的GamePad控制器以及两种单手操作方式进行了对比实验。结果表明,本方案能够带来优秀的沉浸体验,是一种具有一定稳定性的可行方法。

为扩大无人机控制的维度,进一步研究了基于深度学习的动态手势识别技术。我们首先对基于CTC算法的模型在于变长序列上泛化性能进行了可行性分析和不同参数的性能对比分析,看到CNN在短序列(长度2)的识别准确率接近100\%,但在长序列(长度4)上准确率下了约6\%。GRU在比LSTM参数少23\%的前提下,准确率也保持在96\%。Transformer受模型复杂度影响较大,在参数下降90\%后,平均识别准确率也下降了5\%左右。

在此基础上,在同为变长序列识别的先前研究中,对其提出SED-AR模型进行改进,将训练算法和编码算法从自回归改为CTC,在相同参数的条件下进行与上述相同的变长序列泛化性能测试。在长度3数据集上分别实现88\%和97\%的准确率。同时在长度2和长度4的数据上,前者无法正确识别,后者达到约92\%的准确率。以上结果证明了改进的有效性。

\subsection{贡献}

FPV无人机正在快速发展,但针对FPV的无人机远程操作界面研究却进展甚少。本论文基于现有的单手操作研究,提出了结合手腕旋转和手指手势的单手操作模式。通过采用位置映射逻辑,实现了无人机位置的精密操作。与传统GamePad操作的对比实验显示了沉浸感的潜力。与传统IMU操作模式的对比实验中,明确了本方案在稳定性方面的优势。此外,与DJI Avata2单手操作模式的比较中,确认了本方案在沉浸感方面的显著优势。这些实验结果证明了本方案的可行性,为自然手势控制提供了新的可能性。

本研究还提出了更系统的测试路线设计和难度评估方法。这是目前现有研究中的首次尝试。虽然路线难度包含众多要素,但通过提出基本的计算和评估方法,为相关研究提供了一致的评估工具。

在深度学习手势识别方面,本研究首次将序列手势组合应用于无人机控制,避免了传统手势识别中的语义差异问题,具有更好的通用性和泛化性能。通过系统对比四种主流深度学习模型,为边缘设备部署的无人机手势控制提供了完整的技术方案。特别是发现了GRU模型在参数效率和识别性能之间的最佳平衡点,为实际工程应用提供了重要参考。

\subsection{今后的研究}

本方案仍有一些需要解决的问题。首先,需要进一步探索输入(手势)-无人机动作的映射,考虑更直观的手势和映射。为了找到符合人类直觉的手势和映射,需要设计几种不同的方法,通过用户研究比较各自的特点,探索无人机操作时的手部动作习惯。

此外,探索能够与可穿戴设备良好集成的反馈技术也是一个具有挑战性的课题。例如,可以考虑使用振动马达的触觉反馈,或者利用可驱动纺织品、神经刺激技术等来提供反馈的方法。

测试环境也需要进一步改善。需要改进目前使用的无人机模拟器的物理模型,使其更接近实际无人机。例如,应该考虑利用Microsoft提供的AirSim\footnote{https://github.com/microsoft/AirSim}等专业无人机模拟器。此外,调查实际第一人称视角无人机环境中的操作与模拟器环境中操作的差异程度也很重要。因此,需要构建实际的第一人称视角无人机系统,对本方案进行测试。

最后,关于深度学习模型的可用性,仍需要在实际的边缘设备上部署来进一步测试其性能,比如基于TensorflowLite框架在ESP32平台上部署和测试其实时性能。

%---------------------------------------------------------------------------------------------------------------------------
%致谢
\newpage
\section*{致谢}
在进行本研究的过程中,承蒙众多人士的悉心指导和建议,得以完成本论文。
在此谨表达由衷的感谢之意。
特别是给予巨大指导和协助的南京理工大学机械学院的裴荣老师和吴键老师,福冈工业大学信息工学部石原真纪夫老师
担任本论文副审查员的各位老师,以及参与论文撰写的所有人员,在此表示衷心感谢。
同时,对所属研究室的各位成员也再次表示谢意。
%% ---DO NOT MODIFY BEGIN---
\ifx\allfiles\undefined
\end{document}
\fi
%% ---DO NOT MODIFY END---